\section{Introduction}
\label{colors:intro}

% Context
E-graphs are a versatile data structure that is used for various tasks of automated reasoning, including theorem proving and synthesis.
E-graphs have been popularized in compiler optimizations thanks to their ability to support efficient \emph{rewrites} over a large set of terms, while keeping a compact representation of all possible rewrite outcomes.
This mechanism is known as \emph{equality saturation}.
It provides a powerful engine that allows a reasoner to generate all equality consequences of a set of known, universally quantified, equalities.
Possible uses include selecting the best equivalent of an expression according to some desired metric, such as run-time efficiency~\cite{POPL2009:Tate}, size~\cite{flatt2022small,DBLP:conf/fmcad/NotzliBNPRBT22}, or precision~\cite{herbie}
(when used as a compilation phase)
and a generalized form of unification, called e-unification, for application of inference steps (when used for proof search).

In this work we focus on a stepping stone for what we address as \emph{exploratory reasoning}: a range of tasks including all the above optimization procedures, as well as theory exploration \cite{thesy}, rewrite rule inference \cite{ruler}, and proof search \cite{vampire,DBLP:conf/aplas/BrotherstonGP12,cycleq}.
Exploratory reasoning, in general, can be thought of as any reasoning task navigating a large space of potential goals or sub-goals that need to be selected based on some criteria.
Our motivating example comes from TheSy and Ruler, both of which are theory exploration systems based on e-graphs. 
A theory exploration system attempts to both discover and prove mathematical properties from a set of definitions and known lemmas.
Most of the difficulty in theory exploration comes from the generation and filtering of candidates, rather then from the proof procedure itself.
TheSy does so by efficiently filtering a large set of potential conjectures using e-graphs for equality reasoning, and evaluating which should be potentially proved.
While e-graphs are effective for equality reasoning \cite{egg}, handling branching, such as case splitting during proof search, do not have a common solution, and are treated ad-hoc.
For example, a special type of node is introduced in \cite{POPL2009:Tate} to deal with loop conditions, while in \cite{coward2023automating} a special operator is introduced to reason on expressions under certain contexts, and \cite{thesy} creates full copies of the e-graph for each branch being explored.

% Gap
To illustrate this difficulty, we zoom in on an example from theory exploration.
As an example scenario, consider trying to discover and prove lemmas on sorted lists:
a library containing functions $\tlfind$, $\tissorted$, and $\tbinsearch$.
We expect to discover lemmas involving these functions; one such lemma might be the property: $\tissorted\,l \rightarrow \tbinsearch\, l\,v = \tlfind\,l\,v$.
State-of-the-art theory exploration systems~\cite{ITP2017:Johansson,ruler,thesy} have some enumeration strategy over expressions in order to discover candidates. 
A challenge presents itself when some lemmas in the space require an assumption, in this case $\tissorted\,l$.
When dealing with e-graphs,
adding an assumption would \emph{globally}
affect all terms involved in the enumeration, making it impossible to separate conclusions stemming from different assumptions.
Because the system cannot know in advance which assumptions will become relevant for discovering equalities, it is required that it also generate and test multiple candidate assumptions.
An immediate solution is to create one copy of the graph per assumption, but doing so can significantly increase the memory usage.
Moreover, lemmas may depend on one another; for example, $\tissorted\,l \rightarrow \tbinsearch\, l\,v = \tlfind\,l\,v$ depends on transitivity of $\leq$ ($x\leq y \land y\leq z \rightarrow x\leq z$).
Therefore, just trying the candidates one at a time would mean that the system would prematurely discard candidates depending on the order in which they are tested; alternatively, for each candidate that is validated and becomes a lemma, it would be forced to re-try all the previously failed attempts, which is highly costly.

% Theory exploration systems based on random testing  (\todo{speculate, hipspec with conditions}) attempt to greedily discover

% Innovation
To overcome this difficulty, we propose an extension of the e-graph data structure.
An e-graph naturally represents a congruence relation $\cong$, which is an equality relation over terms (with function applications), which  maintains $x \cong y \vdash f(x) \cong f(y)$.
The congruence relation is maintained in the e-graph as a set of equivalence classes (e-classes), which can be merged as part of updating the underlying relation. 
We extend the e-graph data structure into a \emph{Colored E-Graph} to maintain multiple congruence relations at once, where each relation is associated with a color.
Our key observation is that each added assumption, can be treated as a new congruence relation, but is only a coarsening of the original relation. 
The coarsening, then, can be represented as a set of additional merges of e-classes on top of the original e-graph.
The main benefit is reducing memory consumption by re-using and sharing most of the e-classes between colors.
Going back to the sorted list example, in the colored e-graph there will be a \cred relation for assuming $x \le y \land y \le z$, and a \cblue relation for assuming $\tissorted\,l$.
Thanks to the size reduction, multiple relations can exist at once, and thus the lemma $\tissorted\,l \rightarrow \tbinsearch\, l\,v = \tlfind\,l\,v$ can be discovered after transitivity of $\leq$ is proven, but without dependency on the order of exploration.
Colored e-graphs also support having a hierarchy between different colors, which can benefit from additional sharing of e-classes.
For example, the \cred color representing $x \le y \land y \le z$ is itself a coarsening of some \cgreen color representing just the assumption $x \le y$.

While the memory footprint for each color is smaller, maintaining the congruence relation and the data structure invariants becomes more challenging.
To address this we present specialized data-structure modifications and evaluate them.
%We also extend the colored e-graph's logic in several key ways, to support e-graph operations for all the colors.
First, we set up a multi-level union-find where the lowest level corresponds to the root congruence.
Second, we change how congruence closure is applied to the individual congruence relations while taking advantage of the sharing between each such relation and the root. 
Lastly, we present a technique for efficient e-matching over all the relations at once.

% Oh we are so good!
\medskip
Our contributions are:
\begin{enumerate}[leftmargin=1.1em]
    \item The observation that assumptions induce coarsened e-graphs that share much of the original structure. 
    \item Algorithms for colored e-graphs operations.
    \item Optimizations on top of the basic algorithms to significantly improve resource usage.
    \item A colored e-graph implementation, \emph{Easter Egg}\footnote{\url{https://github.com/eytans/egg/tree/features/color_splits}} and an evaluation that shows an improvement factor in memory usage over the existing baseline, while maintaining similar run-time performance.
\end{enumerate}