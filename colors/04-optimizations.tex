\section{Optimizations}
\label{colors:optimizations}

% How we should present this section:
%
%   Intro
%   Show problems that are existing still:
%       a. go into the just presented bad rebuild - go into the e-node count
%       b. inform on worse ematching with duplicate results
%       c. e-node readding
%   sub-section 1 data structure
%   2. Improve data structure to somewhat reduce these problems - Represent colored e-nodes seperately
%       a. Assume small changes again
%       Also Deals with previously presented problems:
%       b. We have a canonized version of colored conclusions to somewhat prevent readding.
%       c. No extra results in e-matching because of seperation.
%       d. Might want to go into big changes complements clones
%   3. We can prune unneeded e-classes in an attempt to improve memory usage for long runs (probably more relevant in cases where some rewrite rules will be used less, to prevent readding).
%   4. A small optimization of having one colored e-class per colored equality class
%   sub-section 2 algorithms
%   Algoritmic improvements that can be made without changes to data structures.
%   5. Improve e-matching by reusing black results as black
%       a. We jump to brothers only on a different "black" path to prevent duplicate paths in the e-graph
%       b. Might be harder to apply conditions
%   6. Improve rebuilding by reusing black conclusions
%   subsection 3 system upgrades
%   As part of the change to colored e-graphs, some other parts of the system might need to change
%   7. e-matching should stream results as we e-match all colors at once
%   8. condition checking needs to actively check all colors for black matches

% Stuff we did not do:
%   1. Parallel colored congruence
%   2. Hierarchical colors
%   3. Had the possibilty of coloring only e-nodes, but that makes the rebuilding process very difficult

\begin{comment}
This section details key optimizations for colored e-graphs, addressing challenges encountered when applying operations such as e-matching and rebuilding.
These optimizations target issues like handling e-nodes across multiple layers, inefficient rebuild processes, duplicate results in e-matching, and e-node re-additions. 
We'll first outline these issues before presenting solutions.    
\end{comment}

Both rebuilding and e-matching in colored e-graph, as discussed in \autoref{colors:overview}, can be significantly slower compared to a separate, minimized e-graph.

In the rebuilding aspect,
two main burdens are that the colored e-graph contains additional e-nodes compared to each of the separate ones, and that building a colored hash-cons (which will be presented shortly) requires going over all the e-classes.

In the e-matching aspect, colored e-matching
may produce duplicate results due to the e-graph not being minimized according to the color's congruence relation;
that is, colored-congruent terms are not always merged under a single e-class id.
To illustrate this, consider a simple e-graph
representing the terms $1\cdot 1$, $1\cdot x$, $1\cdot y$, and $x\cdot y$.
Introduce a color, \cblue, where $x \congblue y$.
A simple pattern such as $1\cdot ?v$ would have three matches, with assignments $?v \mapsto 1, ?v\mapsto x, ?v\mapsto y$.
If the \cblue layer were a separate e-graph, $x$ and $y$ would have been in the same e-class,
so one of the matches here is redundant (as far as the blue layer is concerned).
Of course, in the black layer they are different matches; the point is, that many terms are added to the graph only as a result of a colored match,
so matching them in the black e-graph is mostly useless to the reasoner.
On the other hand, their \emph{presence} in the black layer means they cannot ever be merged, leading to duplicate matches, as seen above, even in the respective colored layer(s).

Moreover, when inserting e-nodes to the e-graph, the hash\-cons is used to prevent duplication, relying on it being canonized.
Adding an e-node from a colored conclusion
(following a match modulo $\congbluesym$) does not benefit from canonization.
In fact, each e-node $f(x_1,\dots,x_n)$ has a multitude of black representatives that are $\congbluesym$-equivalent. 
Each child $x_i$ in the e-node can be presented by any black id such that $e \in [x_i]_b$, so there are $\prod_i |[x_i]_b|$ representations.
These variants are distinct in the root color,
so they cannot be de-duplicated as usual.

To address these issues, we present a series of optimizations to the colored e-graph data-structure and the procedures. 
These optimizations aim to reuse the ``root'' and ancestor layers as much as possible, both in terms of memory usage and compute.
Thus, we can achieve a memory efficient, but effective colored e-graph. 

\subsection{Data-structure optimizations}
% Having the seperate differences
\myparagraph{Colored e-nodes.}
In the basic implementation outlined in \autoref{colors:overview}, adding e-nodes from colored e-matches to the root e-graph may make it very large and increase the cost of all subsequent actions. 
The optimized version addresses this by introducing \emph{colored e-nodes}, where e-nodes resulting from colored matches are tagged with their inducing colors. 
Each colored layer has its own colored hash-cons and e-class map, designed to store only the differences from the parent layer, thereby maximizing reuse.
The new mappings added are:
%\newcommand\eqid{\equiv_{\mathrm{id}}}
%
\[
\renewcommand\arraystretch{1.2}
\begin{array}{ll}
  \mbox{e-class color}& \! \! EC : E \to C  \\
  \mbox{colored parent}& \! \! P_c = \{ (n, e) ~|~ (n, e) \in P \land EC(e) = c \} \\
  \mbox{colored hashcons}& \! \! H_c = \{n \mapsto e ~|~ n \in M(e) \land EC(e) = c \} \\
\end{array}
\]

Note that base parents and hashcons from the non-optimized version are incorporated as $P_\croot$ and $H_\croot$ in colored mappings.

This optimization applies the hierarchy in all operations. 
For example, while inserting an e-node to a color $c$, it is looked up in the colored hash\-cons for $c$ and all its ancestors, $p^*\!(c)$, and finally, if no match is found, it is inserted into a new e-class $e$, setting $EC(e) = c$.
The colored hashcons $H_c$ is canonized to color $c$, ensuring that new e-nodes are unique to this layer and avoiding colored duplicates. 
(Some duplication related to $c$ may still occur in ancestor layers, as their e-nodes are not canonized to $c$.)
The optimization significantly impacts e-matching: previously when matching a function application $f$, all $f$-e-nodes in $N$ were considered; now, only those e-nodes $n$ in the colors hierarchy, that is, those satisfying $\exists e.~n \in M(e) \land EC(e) \in p^*(c)$, are examined.

% Assume small changes
%\SI{why is this paragraph in this particular place}
%The optimized version of the colored e-graphs will have additional e-nodes for each color.
%In the context of equality saturation, any additional colored e-node results from some rewriting conclusion, so the more colored unions happened, the bigger we expect the colored e-graph to be, tying the size of the colored e-graph to the, assumed to be small, changes introduced under each color.

% Present pruning
\myparagraph{Pruning.}
Recall that having a coarsening relation between the colors in the hierarchy means that any result found in an ancestor color is also true for the descendant(s).
And so, following merges, some of the colored e-nodes could become subsumed by e-nodes that already exist in an ancestor layer.
We present an efficient deferred pruning method to remove the redundant e-nodes.

Normal e-graph minimization relies on having all e-nodes canonized.
A colored e-graph usually does not canonize all e-nodes to a specific color $c$ (except for \croot). 
Rather, $H_c$ contains only the difference from previous layers. 
To find redundant e-nodes, the colored e-graph  builds a transient hashcons during rebuild from all relevant e-nodes that are not $c$-colored.
The new hashcons, $H'_c$, is created as follows:
\[
\renewcommand\arraystretch{1.2}
\begin{array}{l@{}l}
H'_c = \{& \mathit{canonize}_c(n) \mapsto \tfind_c(e) ~|~ \\
& ~~n \in M(e), EC(e) \in p^+(c) \}
\end{array}
\]
% redundant = $\{ n | \exists e. n \in M(e) \land EC(e) = c \land n \in H'_c \}$
A $c$-colored class $e$ can be reduced by removing all e-nodes that already exist in $H'_c$.
While pruning is promising, one must take care that pruned e-nodes are not immediately re-added.
% We can add that this is probably when we add conclusions back to the black layer or we stop using certain rewrite rules.

% Present the "single colored e-class per colroed equality class" thing
\myparagraph{Colored minimization.}
Another improvement is having multiple colored e-nodes (of the same color) in a single (black) e-class. 
As mentioned previously, any e-node that resulted from a colored insert had to be in their own e-classes, as no black unions would be performed on them.
But, given that $e \equiv_c e' \land EC(e) = EC(e') = c$, then the two black e-classes $e,~e'$ can be merged as both contain colored e-nodes of the same color and are in the same colored e-class (of the same color).
Thus an invariant is kept that each colored equality class has at most one black e-class containing colored e-nodes.

\subsection{Procedure optimizations}
%   6. Improve rebuilding by reusing black conclusions
\myparagraph{Rebuild.}
When rebuilding, we first reconstruct the congruence relation of the ``root'' layer.
Even though a color, for example {\color{blue}blue}, will need to rebuild its own congruence, it still holds that ${\cong}\subseteq{\congblue}$.
So, any union induced by ${\cong}$ can be applied to the {\color{blue}blue} relation.
To understand the implications, consider the e-graph representing the terms $x$, $y$, $f(x)$, $f(y)$, $f(f(x))$, and $f(g(y))$ where the {\color{blue}blue} color contains the additional assumption that $g(y) \congblue f(y)$.
If we union $x$ and $y$, the black congruence will include $f(x) \cong f(y)$  which also holds in the blue relation.
But, the rebuilding of the blue congruence invariant will include an additional, deeper (in terms of rebuilding rounds), conclusion $f(f(x)) \congblue f(g(y))$.
This demonstrates how reusing parent relations is useful; the rebuild depth can be reduced by first rebuilding finer relations.

%   5. Improve e-matching by reusing black results as black
%       a. We jump to brothers only on a different "black" path to prevent duplicate paths in the e-graph
\myparagraph{E-match.}
In e-matching, we implement an optimization where findings on the root layer are also valid for higher layers. 
To avoid redundant pattern matching, e-matching begins only from \croot, adding colored assumptions as needed. 
There are two scenarios for introducing a colored assumption: 
The first during \emph{compare($i, j$)}, if $\treg[i] \not \equiv_{\tcolor} \treg[j]$, we explore descendant colors $c$ where $\treg[i] \equiv_c \treg[j]$, adding states with $\tcolor \gets c$ to the backtracking stack $bs$.
The second is on-demand coloring in \iclrjmp, where jumps to any color $c$ are enabled if $M.color \in p^+(c)$ and the target e-class is otherwise unreachable.
We minimize the set of new assumptions to prevent redundant colors. 
During the updated \icompare, \icompare', if a color $c$ is sufficient, its descendants are not added to $bs$. 
For to updated \iclrjmp, \iclrjmp', e-classes are matched only with their topmost (closest to root) congruent descendants. 
By taking the topmost descendants, we ensure that all additional matching paths are unique, as at least one (different) e-class is chosen at each fork.
Despite eliminating duplicate paths, some duplicate colored matches persist due to incomplete minimization of the e-graph.
The modified instructions are described in more detail in \annexref{app:algorithms}. 

\begin{comment}
    To support matching multiple colors at once, we modify the machine such that it will add colored assumptions on demand during the search, and push it into the backtracking stack $bs$.
    For that purpose, we modify $M$ to also include the current colored assumption $color$, which will be initialized to $\varnothing$.
    There are two situations where a colored assumption will be added.
    The first is during \emph{compare($i, j$)}.
    If $reg[i] \neq reg[j]$, then we don't necessarily need to backtrack, as there might be a descendant of $color$, $c$, under which $reg[i] \equiv_c reg[j]$.
    Therefore, for each such $c$ the current state with the new assumption $color \gets c$ is add to the backtracking stack $bs$.
    The second case is when matching a function application of $p$ which is represented by a bind instruction.
    Before each bind instruction, a modified compilation procedure will insert a new \emph{colored\_jump} instruction.
    The \emph{colored\_jump($i$)} instruction will add all the "colored siblings" of the current $color$ to $bs$.
    In addition any additional colored assumption $color`$ that is descendant of $color$, and has additional "colored siblings" will also be pushed to $bs$.
    The algorithms for \emph{colored\_jump($i$)} and the updated \emph{compare($i, j$)} are given in \ES{TODO}.
\end{comment}


%       b. Might be harder to apply conditions

%  TODO: I think we should not include system upgrades.
% subsection 3 system upgrades
%   As part of the change to colored e-graphs, some other parts of the system might need to change
%   7. e-matching should stream results as we e-match all colors at once
%   8. condition checking needs to actively check all colors for black matches



% Therefore, TheSy does not keep all the assumptions (E-Graph clones). Rather, TheSy uses the case splitting mechanism to find relevant equalities only at a few strategic points in the exploration, and then frees all the clones to save memory.
% \ES{smoother}
% We aim to further amortize costs while enhancing the memory usage efficiency of the E-Graph, by maximizing component reuse. 
% This approach requires a detailed understanding of the primary data structures employed by the egg library. 

% Remind use case and go over the problems we need to deal with when no colored e-nodes
\begin{comment}
Colored e-graphs are designed to be used in larger scale test cases where some tool is using the e-graph for exploration tasks.
This means that the e-graph will contain numerous e-nodes, and possibly multiple colors.
While we hope the reader is convinced our e-node sharing approach is useful, the solution provided so far will be inapplicable to any real case.
The main culprit is that we insert all e-nodes to the black relation.
Having all e-nodes exists in the black relation means they will also exist for all colored relations.
While this doesn't seem detrimental to the performance at first, the problem actually arises from having additional successful applications of rewrites.
Of course it seems like a good thing because each application retains soundness, and we gain more conclusions in the e-graph.
But, having all these conclusions means that even for each additional e-node we might attempt to match it in additional rewrites, and the rebuilding will need to go over it.
Also, rebuilding for colored congruence relations, which is inefficient, will need to be applied to all e-nodes of all colors.
\end{comment}

% A good example: For optimizations: For example, in a relation where all E-Classes are unioned, E-Matching will be very inefficient due to the caveat we presented.
% This issue becomes more pronounced when considering the additional results in e-matching. As the number of successful rewrite applications increases, more e-nodes are added to the e-graph. This may seem beneficial, as each application maintains soundness and results in more conclusions within the e-graph. However, in our experiments [TODO], we do not observe any of these advantages.

