\section{Conclusion}
\label{colors:conclusions}
% We created colored e-graphs for multiple congruence relations
% We evaluated and it is good


% We want future work


\begin{comment}
In conclusion, this paper has introduced the concept of colored e-graphs as a memory-efficient method for maintaining multiple congruence relations in a single e-graph. 
It provides support for equality saturation with additional assumptions over e-graphs, thereby enabling efficient exploratory reasoning of multiple assumptions simultaneously.
The development of several optimizations based on the egg library and deferred rebuilding, and subsequent evaluation has validated our approach, demonstrating a significant improvement in memory utilization and a modest one for run-time performance compared to the baseline.
\end{comment}

We presented colored e-graphs as an approach to efficiently handle multiple congruence relations in a single e-graph.
They provide a memory-efficient method for equality saturation with additional assumptions, crucial for efficient exploratory reasoning of multiple assumptions simultaneously.
Our optimizations, developed using the egg library, have shown notable improvements in memory usage and moderate enhancements in run-time performance over the baseline.

%This work thus serves as a stepping stone, advancing the current state of the art and setting a foundation for works on exploratory reasoning tools and techniques. 
%By extending e-graph capabilities, we hope to drive new innovation in the realm of symbolic reasoning and its applications.