\section{Functional Description}
\label{colors:functional}

We now introduce some notations and definitions that formalize
the description of the e-graph presented in \autoref{colors:overview}.
We assume a term language $L$ where terms are constructed using \emph{function symbols}, each with its designated arity.
We use $f^{(r)}\in\Sigma[L]$ to say that $f$ is
in the \emph{signature} of $L$ and has arity $r$.
A term is then a \emph{tree} whose nodes are labeled by function symbols and a node labeled by $f$ has $r$ children.
(In particular, the leaves of a term have nullary function symbols.)
Additionally we use the following definitions:
%
\[
\renewcommand\arraystretch{1.2}
\begin{array}{ll}
  \mbox{e-class ids} & E  \\
  \mbox{e-nodes}  & N = \{f(e_1,..,e_r)~|~ f^r\in\Sigma, e_i\in E\} \\
  \mbox{union-find} & {\eqid} \subseteq E\times E, \quad \mbox{${\eqid}$ is an equivalence relation} \\
  \mbox{e-class map} & M : E \to \mathcal{P}(N) \\
  \mbox{parent map}  & P = \{e \mapsto \{ (n, e') ~|~ e' \in E~\land \\
  & \quad n \in M(e') \land n = f(\ldots, e, \ldots) \} ~|~ e \in E \} \\
  \mbox{hashcons} & H = \{n\mapsto e ~|~ n\in M(e)\} \\
\end{array}
\]

Semantically, every e-class represents a set of terms over $\Sigma$.
We will use the notation $[t]$ to refer to e-class id of the equality class that represents (among other terms), the term $t$.

The union-find structure offers an operation, $\tfind(e)$, that returns a unique representative id of the equivalence class (of $\eqid$) that contains $e$.
That is, $\tfind(e) \eqid e$ and for all $e_1 \eqid e_2$, $\tfind(e_1) = \tfind(e_2)$.

On top of these basic structures, we introduce a set of \emph{colors}. 
As explained in \autoref{colors:overview}, colors are organized in a tree whose root is the initial color (``black'').
We mark the root color $\croot$
and assign to every non-root color $c$ a \emph{parent color} $p(c)$.
%
\begin{comment}
\[
\begin{array}{ll@{~~~}l}
\mbox{color tree} & C = \{\croot, \ldots\}
& p : C \setminus \{\croot\} \to C
\end{array}
\]
\end{comment}
\[
\renewcommand\arraystretch{1.2}
\begin{array}{ll}
  \mbox{colors} & C = \{\croot, \ldots\}  \\
  \mbox{parent colors} & p : C \setminus \{\croot\} \to C
\end{array}\hspace{2cm}
\]

\smallskip
The colored e-graph will now hold multiple union-find structures, one per color.
They define a family of equivalence relations $\equiv_{c}$ by induction
on the path from $\croot$ to $c$.

\begin{itemize}
    \item ${\equiv_{\varnothing}} = {\eqid}$~;\quad $\tfind_{\varnothing}(e) = \tfind(e)$
    \item ${\equiv_c} \subseteq E_{p(c)}\times E_{p(c)}$, where
      $E_{p(c)}=\{\tfind_{p(c)}(e)~|~e\in E\}$ is the set of all representatives from $\equiv_{p(c)}$. $\tfind_c\hspace{-0.5pt}(e)$ for $e\in E_{p(c)}$ returns a unique identifier in the normal manner of union-find, \ie, $\tfind_c\hspace{-0.5pt}(e) \equiv_c e$  and for all $e_1 \equiv_c e_2$, $\tfind_c\hspace{-0.5pt}(e_1) = \tfind_c\hspace{-0.5pt}(e_2)$.
\end{itemize}


\begin{comment}
Formalize the colored-rebuild algorithm
This should include an algorithmic description of normal rebuild
Then we would use functions to use the same algorithm for colored rebuild.
The main concerns that change are:
1. Collecting all the parents
2. Updating the memo
\end{comment}

The definitions over $E_{p(c)}$ are naturally extended to $E$ by (recursive) application of $\tfind$; i.e., $\tfind_c(e) = \tfind_c(\tfind_{p(c)}(e))$
and $e_1 \equiv_c e_2 \Leftrightarrow \tfind_{p(c)}(e_1) \equiv_c \tfind_{p(c)}(e_2)$.
Thus it holds, by construction, that
${\equiv_{c}} \supseteq {\equiv_{p(c)}}$.

The colored e-graph also supports a $\tmerge_c(e_1, e_2)$ operation for each color $c$ where $e_1, e_2 \in E_c$.
The merge operation may break the congruence relation invariants for $c$ and all its descendants, and thus needs to be fixed.
The merged classes are added to $\mathit{worklist}(c')$ for all $c'$ where $c'$ is $c$ or one of its descendant.
In egg~\cite{egg}, the invariants are restored periodically by performing a \textsc{rebuild} pass.
To accommodate the colors, we adjust the
\textsc{rebuild} logic to a multi-congruence-relation setting,
so that it restores a congruence closure for each color during \textsc{rebuild}.
The main difference is that for a colored congruence relation, the procedure will collect the parents of a colored e-class by combining the sets of parents of all the (root) e-classes contained therein.

\begin{comment}
We update the auxiliary function \textsc{repair}
to work on colored e-classes,
and introduce two new helper functions: $\textsc{collect\_parents}$ and $\textsc{update\_hashcons}$, as presented in \autoref{functional:repair}.
$\textsc{collect\_parents}$ extract the parents of a colored e-class by combining the sets of parents of all the (root) e-classes contained therein.
$\textsc{update\_hashcons}$ is used to make sure that the hashcons entries are in canonical forms. It was already a part of \textsc{repair} in egg;
it is only repeated here to point out that it
only updates the hashcons for the root color,
since no canonization is required for colored layers.    
\end{comment}

Another important colored e-graph operation is e-matching.
Colored e-matching is a modification of the e-matching abstract machine presented in \cite{DBLP:conf/cade/MouraB07}.
E-matching is performed by an abstract machine $M$ which consists of a program counter, array of registers $reg$, and backtracking stack $bs$, in combination with a sequence of instructions that represents a pattern $p$. 
The machine will run instructions by order, where each may either fail if its assertion is not met, or produce a set of continuation states.
If a continuation state is produced, the machine selects the first one and adds the current instruction to the stack. 
If no continuation state is produced, the machine backtracks, retrieving the most recent state from the stack and attempting the next available continuation.

To better present our modifications in colored egg, we first shortly introduce some of the original instruction types:
\begin{itemize}[leftmargin=1.1em]
    %\item $\iinit(f)$ --- Expects an e-node representing an application $f(x_1,\ldots,x_n)$ in $\treg[0]$, and pushes its children into $reg[1..n]$.
    \item $\ibind(\tin, f, \tout)$ --- Matches any e-node of the form $f(x_1,\ldots,x_n)$ that resides in the e-class saved in $reg[in]$, storing its children $x_{1..n}$ in $reg[out..out+n-1]$.
    \item $\icompare(i, j)$ --- Asserts $\treg[i] == \treg[j]$.
    \item $\icheck(i, term)$ --- Asserts that the e-class $\treg[i]$ represents $\tterm$.
    \item $\icontinue(f, out)$ --- Match any e-node 
    $f(x_1,\ldots,x_n)$ (in \emph{any} e-class),
    storing its children $x_{1..n}$ in $\treg[\tout..\tout+n-1]$.
    \item $\ijoin(\tin, \trevpath, \tout)$ --- Match any e-node
    $f(x_1,\ldots,x_n)$ that is reachable through $\trevpath$
    from the e-class $\treg[\tin]$,
    storing its children $x_{1..n}$ in $\treg[\tout..\tout+n-1]$.
\end{itemize}

To facilitate matching across various congruence relations, we adjust the machine $M$ to include the, currently being e-matched, colored assumption $color$ in its state.
Adapting to $color$ involves changes in compilation and instructions. 
The two primary scenarios impacted are: during $\icompare(i, j)$, ensuring $\treg[i] \equiv_{color} \treg[j]$, and in function application matching represented by a \ibind~ instruction.
Before each `\ibind' instruction, the modified compilation will insert a new `\iclrjmp' instruction to try matching the full colored equality class, one ``root'' e-class at a time.
This is achieved by having `$\iclrjmp(i)$' yield all the ``colored siblings'' of $\treg[i]$ in the current $\tcolor$, replacing $\treg[i]$ with the result.
%The algorithms for \textsc{colored\_jump}($i$) and the updated \textsc{compare}($i, j$) are given in \autoref{functional:colored_machine}.
The instruction `check' can be likewise adjusted, but we point out
that, in fact, it can be implemented as a sequence of `\ibind's (with respective interleaved `\iclrjmp's).

Multipatterns, supported by the abstract machine, enable e-matching against patterns with shared variables, useful for matching the precondition in conditional rewrite rules.
This is achieved using the `\icontinue' instruction, which selects a new root for subsequent sub-patterns. 
In the colored setting, while `\icontinue' remains as is, for performance, it's sometimes substituted with `\ijoin'. 
This alternative instruction also picks a new root, but restricts selection to e-nodes that can reach a specified e-class, linked to a previously matched hole, through child edges in the e-graph.
A \emph{reverse path} is provided to further restrict the upward search needed to find such e-nodes.
We do not go too deep into the details, but its colored variant
will invoke a $\iclrjmp$ at every level.
We point out that egg does not currently implement `\ijoin',
and our colored egg supports a special (though frequent) case in which
$\trevpath$ is empty.

The algorithms described here are presented in more depth in \annexref{app:algorithms}.