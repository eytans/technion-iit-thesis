\section{Future Work}

\TheSy has been built from the grounds up based on term rewriting, and the results s what previous, non-symbolic tools have been able to accomplish.
However, it is our opinion that the advantages lie in its  potential to extend to new domains that cannot be
handled by concrete testing and SMT solvers.

Much of the appeal of \TheSy as a new technique for theory
exploration is its versatility in handling abstract values.
Since concrete data elements are not required for the lemma vetting process,
\TheSy can be extended to arbitrary families of types.
Our experiments have shown that it successfully handles first-class functions without generating the function bodies, but based solely on their signatures.
The next step would be to add support for refinement types~\cite{pldi91/freeman},
and notably, dependent types.
These were shown to be excellent tools in reasoning about the correctness of software~\cite{icfp14/vazou,book/cpdt,oopsla19/jad}.
In particular, their combination --- dependent refinement types --- is most interesting because it exposes the tight interconnection between types and propositions.
This makes it an ideal challenge for symbolic, constructive reasoning.

A second pain point, exposed by our experiments, is handling case splits when comparing terms as well as by carrying out proofs.
Testing technology has evolved several tools targeted specifically at handling conditional control flow in over 40 years of research in the field (\cite{cacm76/king}, with some commercial outcomes such as \cite{TaP2008:Tillman}).
The appeal of applying a proof-theoretic approach is its ability to employ mathematical abstractions to describe classes of values.
The current implementation of \TheSy essentially forks execution when splitting by case, copying the entire state.
A more careful treatment can distinguish facts that are case-specific from global ones, making the search more focused and allowing deeper proof exploration.