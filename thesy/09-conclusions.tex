\section{Conclusion}
\label{thesy:conclusions}

We described a new method for theory exploration, which differentiates itself from existing work by basing the reasoning on a novel engine based on term rewriting.
The new approach differs from previous work, specifically those based on testing techniques, in that:
\vspace{-.25em}
\begin{enumerate}
    \item This lightweight reasoning is purely symbolic, supporting value abstraction and performs better then prior art.
    \item Functions are naturally treated as first-class objects, without specific support implementation.
    \item The only needed input is the code defining the functions involved, and no support code such as a specific theory solver or random value generators.
    \item \TheSy has a unique feedback loop between the prover and the synthesizer, allowing more conjectures to be found and proofs to succeed.
\end{enumerate}
\vspace{-.25em}

By creating a feedback loop between the four different phases, term generation, conjecture inference, conjecture screening and induction prover, this system manages to efficiently explore many theories.
This goes beyond similar feedback loops in existing tools, aiming to reduce false and duplicate conjectures.
As explained in \autoref{overview:conjecture}, this form is also present in \TheSy, but \TheSy utilizes this feedback in more phases of the computation.

Theory exploration carries practical significance to many automated reasoning tasks,
especially in formal methods, verification and optimization.
 Complex properties lead to an ever-growing number of definitions and associated lemmas, which constitute an integral part of proof construction.
These lemmas can be used for SMT solving, automated and interactive theorem proving, and as a basis for equivalence reduction in enumerative synthesis.
The term rewriting-based method that we presented in this paper is simple, highly flexible, and has already shown results surpassing existing exploration methods.
%it is even competitive with state-of-the-art provers based on SMT technology.
The generated lemmas allow even this simple method to prove conjectures that normally require sophisticated SMT extensions.
Our main conclusion is that deductive techniques and symbolic evaluation can greatly contribute to theory exploration, in addition to their existing applications in invariant and auxiliary conjecture inference.

\subsubsection*{Acknowledgements.}
\label{sec:conclusion}

This research was supported by the Israeli Science Foundation (ISF) Grants No.~243/19 and
2740/19 and by the United States-Israel Binational Science Foundation (BSF) Grant
No.~2018675.