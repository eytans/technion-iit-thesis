\section{Introduction}

Proof assistants have become increasingly important tools in formalization of mathematics, formal verification, and program analysis. 
The have become indispensable in tasks such as verification of complex systems, that require a high level of expressivity \cite{bedrock, compcert}.
However, their usability depends on an expert user being able to select the appropriate proof steps within the interactive environment.
Users can take advantage of automation features provided by the proof assistant, but this requires a higher level of skill~\cite{qedatlarge}.

Researchers recently had substantial success in using automated theorem proving (ATP) as a component in interactive theorem provers (ITPs, often referred to as proof assistants).
An ATP can be a useful automatic one click solution to the vast amount of effort put into verifying software using ITPs \cite{qedatlarge,compcert,bedrock}.
However, ATPs usually reason over first order logic (FOL), and proof assistants use a more expressive high order logic (HOL), requiring an encoding of the proof assistants environment into FOL.
This is usually done by tools called \emph{hammers}, and for the ATPs to be effective, requires facts (axioms and lemmas) selection mechanism, as too many axioms will often make the proof search diverge \cite{coqhammer, hammeringqed, sledgehammer}.

Despite their effectiveness in certain scenarios, hammers present significant limitations in terms of interactivity within proof assistants.
A key issue that when a hammer fails to prove a theorem, it offers no insights into why the proof attempt was unsuccessful or how the user might proceed. 
This lack of intermediate feedback is particularly problematic in interactive theorem proving environments, where users often require guidance to refine their proof strategies or identify missing lemmas.
This limitation restricts the ability of expert users to effectively guide the proof process in complex scenarios where automated methods alone are insufficient.

\begin{comment}
This chapter introduces an approach dubbed Lightweight Equality Saturation (LES) for proof assistants, which aims to apply equality saturation to a reduced form of the available theory, thus maintaining computational efficiency required for an interactive setting.
By selectively applying saturation techniques to critical subterms and leveraging domain-specific heuristics, LES offers a promising framework for enhancing the capabilities of proof assistants without incurring prohibitive performance costs.
\end{comment}

This chapter introduces an approach called Lightweight Equality Saturation (LES) for proof assistants, which aims to somewhat bridge the gap between automated theorem proving and interactive proof environments. 
LES leverages the power of equality saturation \cite{eqsat}, a technique to effectively reason about equality of terms, through an the ITPs hammer mechanisms.
Equality saturation operates by repeatedly, and simultaneously, applying rewrite rules on top of an e-graph \cite{egg}, a data structure that compactly represents equivalence classes of expressions. 
By employing a smart encoding to an e-graph and carefully constructed rewrite rules, LES achieves a balance between speed and effectiveness as a prover.

A key advantage of the LES approach is its ability to potentially provide meaningful feedback even when it fails to prove a theorem completely. 
In such cases, LES can attempt to offer a simplified sub-goal that, if proven, would enable the proof of the full goal. 
This can significantly enhances the interactive experience, allowing users to make incremental progress and gain insights into the proof structure. 
It's important to note that an eqsat-based prover like LES has inherent limitations: it struggles with existential quantification and lacks built-in mechanisms for case splitting or induction. 
However, these limitations are well-suited to the ITP scenario, where a rich ecosystem of existing lemmas often already addresses case reasoning or induction steps.
To ensure successful application and maintain efficiency, the LES translation process incorporates several crucial optimizations. 
Given that e-graphs are sensitive to size due to the exponential nature of e-matching (the basis for rewriting in equality saturation), the translation carefully avoids introducing unnecessary terms into the e-graph. 
Furthermore, LES implements special refinements for handling equality, if-and-only-if statements, and propositions, making it particularly well-suited for pure type systems commonly used in proof assistants.
But, to keep the prover fast, no attempts to deal with existentials or conditionals are made.
Still, the prover manages to go toe-to-toe with state of the art theorem provers, and improve CoqHammer's overall results.

The main contributions of this work are threefold. 
\begin{enumerate}
    \item We present LES, a novel implementation of Lightweight Equality Saturation for Coq.
    \item We introduce a tool that leverages Coq-LSP, a language server protocol meant to replace \cite{serapi} to facilitate the easy creation of benchmarks. 
    This tool significantly streamlines the process of generating and managing test cases for any coq tactic, enabling more comprehensive and efficient evaluation of proof automation techniques.
    \item We provide an evaluation of LES on the MathComp coq library, comparing its performance against existing automated reasoning tools, cvc4 \cite{cvc4}, vampire \cite{vampire}, z3 \cite{z3}, and the E prover \cite{eprover}, and assessing its impact on CoqHammer \cite{coqhammer}. 
    This evaluation demonstrates the effectiveness of our approach. Also an ablation study is done to asses the effectiveness of a critical encoding design choice on the effectiveness of the LES prover.
\end{enumerate}  

% Even when solvable by tactics, requires experts
% This is expensive because assholes like Eytan ask for big salaries
% Automatic tools that will allow non-experts to handle this french piece of shit, often fail, and provide no help