\section{Evaluation}
\label{les:eval}

We evaluate our approach by implementing it as an additional prover in CoqHammer, and run all the provers CoqHammer works with on a benchmark created from the MathComp library.
For that purpose, we present a new tool to create benchmarks for Coq tactics and tools.
Our tool is based on Coq-LSP which is a language server protocol implementation and a continuation of SerAPI \cite{serapi}.
The tools main advantage is stability, as it is a simple plugin to Coq-LSP, it will work as long as Coq-LSP can compile the coq file. 
When run, it will run the tactic under check on each proof goal available, simplifying the creation of new benchmarks to the simple act of running the tactic.

We use a built in mode from CoqHammer to run all translation options for each proof goal into a TPTP file \cite{sutcliffe2017tptp}, where the options are different ways to perform premise selection.
We then run each of the provers on all translated goals.

All the experiments were conducted on 64 core AMD EPYC 7742 processor with 512 GB RAM.

\subsection{Implementation Optimizations}

To ensure practical applicability, LES incorporates several key optimizations:
Constant Axiom Filtering: Large constant axioms are filtered based on size before addition to the e-graph. This prevents size blowup from single, expansive axioms (e.g., 100 + 1 = 101) that could lead to numerous spurious matches and significantly slow down the proving process.
Parallel E-matching: LES implements parallel e-matching in the underlying e-graph library (egg). Given that e-matching is typically the most time-consuming operation in equality saturation, this parallelization can dramatically reduce overall runtime.
Rewrite Rule Search Limitation: To prevent potentially infinite or excessively long searches, LES imposes a limit on the number of terms each rewrite rule can explore before giving up. While this might cause some proofs to fail, it is crucial for preventing pathological cases where bad rewrites dominate the computation time as the e-graph grows.
These optimizations collectively enable LES to handle a wider range of problems efficiently, making it more suitable for integration into interactive proving workflows where responsiveness is key.
