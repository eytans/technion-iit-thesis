% Use this file to create "glossary entries" for abbreviations and acronyms,
% and other notation. The entries defined here don't necessarily have to be 
% used in the thesis (but then you have to decide whether or not to display
% the unused entries).

% For this file to compile (and the example text in the main/prelims.tex file),
% the package glossaries-extra is required. It is automatically included unless
% the noabbrevs class option is used.

% The following will alter the style for typesetting abbreviations when using 
% the \gls command. Note you can also use multiple styles by categorizing 
% abbreviations; see the documentation for the glossaries-extras package at:
% https://ctan.org/pkg/glossaries-extra
%
%\setabbreviationstyle[acronym]{long-short-sc}
%
% If you're wondering why we're setting the seemingly-redundant "notation 
% category", that's a hack discussed here:
% https://tex.stackexchange.com/q/630541/5640 

\newacronym[%
  category=notation-category,%
  description=``The Senate and People of Rome'']% The description does not appear anywhere by default
  {spqr}% the key of the acronym (used with the \gls command for example)
  {SPQR}% the short form of the acronym
  {Senātus Populusque Rōmānus}% the long form of the acronym

\newacronym[%
  category=notation-category,%
description=A technology used in data storage devices]%
  {smart}{SMART}{Self-Monitoring, Analysis and Reporting Technology}

\newacronym[%
  category=notation-category,%
  description=to build or produce something{,} rather than purchasing it ready-made or paying someone to make it]%
  {DIY}{DIY}{do it yourself}

\newacronym[%
  category=notation-category,%
  description=a four-letter acronym]
  {etla}
  {ETLA}
  {extended three-letter acronym}

\newabbreviation[%
  type=notation,%
  category=notation-category,%
  description=]% This abbreviation has no description; only the abbreviation and the unabbreviated form will be shown
  {aut}{Aut}{Automorphism group}

\newglossaryentry{symb:c}{%
  type=notation,%
  category=notation-category,%
  name=$c$,%
  description=the speed of light%
}

\newglossaryentry{symb:a-b-closed}{%
  type=notation,%
  category=notation-category,%
  name=\ensuremath{a \pm b},%
  description=the closed interval \ensuremath{\left[a-b,a+b\right]}%
}

\newglossaryentry{supercali}{%
  type=notation,%
  category=notation-category,%
  name=supercalifragilisticexpialidocious,
  description=%
    Atoning for being educable through delicate beauty;
    Something to say when you have nothing to say.}

% --------------------------------

% Commands below will control the behavior/appearance of the list of abbreviations and acronyms

% Uncomment this command to have _all_ abbreviations and acronyms defined
% in this file appear in the final list - rather than just the ones you
% use in the thesis
%\keepUnusedAbbreviations
\newabbreviation[%
  type=notation,
  category=notation-category,%
  description={Automated reasoning technique that compactly represents sets of equivalent expressions}]%
  {eqsat}{EqSat}{Equality Saturation}

\newabbreviation[%
  type=notation,
  category=notation-category,%
  description={Data structure that efficiently represents sets of equivalent expressions}]%
  {egraphs}{e-graphs}{Equality Graphs}

\newabbreviation[%
  type=notation,
  category=notation-category,%
  description={Equality Classes}]%
  {eclasses}{e-classes}{Equality Classes}

\newabbreviation[%
  type=notation,
  category=notation-category,%
  description={Theory exploration system leveraging equality saturation}]%
  {thesy}{TheSy}{Theory Synthesizer}

\newabbreviation[%
  type=notation,
  category=notation-category,%
  description={Extension to e-graphs for efficient conditional reasoning}]%
  {cegraph}{Colored E-graph}{Colored Equality Graph}

\newacronym[%
  category=notation-category,%
  description={Equality saturation-based automated reasoning tool for proof assistants}]%
  {les}{LES}{\emph{Lightweight Equality Saturation}}

\newacronym[%
  category=notation-category,%
  description={Interactive Theorem Prover}]%
  {itp}{ITP}{Interactive Theorem Prover}

\newacronym[%
  category=notation-category,%
  description={Automated Theorem Prover}]%
  {atp}{ATP}{Automated Theorem Prover}

\newacronym[%
  category=notation-category,%
  description={First-Order Logic}]%
  {fol}{FOL}{First-Order Logic}

\newacronym[%
  category=notation-category,%
  description={Higher-Order Logic}]%
  {hol}{HOL}{Higher-Order Logic}

\newacronym[%
  category=notation-category,%
  description={Calculus of Inductive Constructions}]%
  {cic}{CIC}{Calculus of Inductive Constructions}

\newacronym[%
  category=notation-category,%
  description={Pure Type System}]%
  {pts}{PTS}{Pure Type System}

\newacronym[%
  category=notation-category,%
  description={Proof Irrelevant Pure Type System}]%
  {pipts}{piPTS}{Proof Irrelevant Pure Type System}

\newacronym[%
  category=notation-category,%
  description={Equality-Program Expression Graphs}]%
  {epeg}{E-PEG}{Equality-Program Expression Graphs}

\newglossaryentry{symb:congruence}{%
  type=notation,%
  category=notation-category,%
  name=\ensuremath{\sim},%
  description={Congruence relation in e-graphs}%
}

\newglossaryentry{symb:forall}{%
  type=notation,%
  category=notation-category,%
  name=\ensuremath{\forall},%
  description={Universal quantification}%
}

\newglossaryentry{symb:exists}{%
  type=notation,%
  category=notation-category,%
  name=\ensuremath{\exists},%
  description={Existential quantification}%
}

\newglossaryentry{symb:implies}{%
  type=notation,%
  category=notation-category,%
  name=\ensuremath{\Rightarrow},%
  description={Logical implication}%
}

\newglossaryentry{symb:iff}{%
  type=notation,%
  category=notation-category,%
  name=\ensuremath{\Leftrightarrow},%
  description={If and only if (logical equivalence)}%
}

\newglossaryentry{symb:eclass}{%
  type=notation,%
  category=notation-category,%
  name=\ensuremath{[e]},%
  description={The equality class (e-class) containing the expression $e$ in an e-graph. It represents all expressions that are known to be equivalent to $e$}%
}

\newglossaryentry{rewrite-rule}{%
  type=notation,%
  category=notation-category,%
  name={\ensuremath{\mathcal{R} = t_1 \overset{.\,}{\rightarrow} t_2}},%
  symbol={\ensuremath{\mathcal{R} = t_1 \rwto t_2}},%
  description={A rewrite rule $\mathcal{R}$ representing a universally quantified formula of the form $\forall x_1, \ldots, x_n.~t_1 \implies t_2$,
    and all free variables are implicitly quantified.}%
}

\newglossaryentry{context-filling}{%
  type=notation,%
  category=notation-category,%
  name={\ensuremath{C[t]}},%
  description={Context filling operation, where $C$ is a context (a term with a single hole) and $t$ is the term used to fill the hole}%
}

\newglossaryentry{substitution}{%
  type=notation,%
  category=notation-category,%
  name={\ensuremath{t\sigma}},%
  description={Substitution operation, where $\sigma$ is a mapping from variables to terms, applied to replace variables in the term $t$}%
}

\newglossaryentry{metavariable}{%
  type=notation,%
  category=notation-category,%
  name={\ensuremath{?v_i}},%
  description={A metavariable, represented as $?v_i$, is used as a hole in a pattern. 
  A pattern is a term with zero or more such holes.}%
}

\newacronym[%
  category=notation-category,%
  description={Rules that specify how to transform one term into another equivalent term in a formal system.}]%
  {rr}{r.r.}{\emph{rewrite rules}}

  \newacronym[%
  category=notation-category,%
  description={A rewrite rule that includes a precondition. It takes the form $\phi \Rightarrow t_1 \rwto t_2$, where $\phi$ is a condition that must be satisfied for the rule to be applied.}]%
  {crr}{c.r.r.}{\emph{conditional rewrite rule}}

  \newacronym[%
  category=notation-category,%
  description={Symbolic Observational Equivalence}]%
  {soe}{SOE}{\emph{Symbolic Observational Equivalence}}

\newacronym[%
  category=notation-category,%
  description={A composite data type defined by specifying a collection of constructors, each with its own arguments.}]%
  {adt}{ADT}{Algebraic Data Type}

\newacronym[%
  category=notation-category,%
  description={The use of computer systems to assist in the creation, modification, analysis, or optimization of a design.}]%
  {cad}{CAD}{Computer-Aided Design}