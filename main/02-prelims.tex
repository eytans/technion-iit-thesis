\chapter{Preliminaries}
\label{chap:prelims}

This chapter introduces the fundamental concepts of rewriting systems, e-graphs, and equality saturation, which form the foundation for the contributions presented in this thesis. 
For this chapter we assume a term language $L$ where terms are constructed using \emph{function symbols}, each with its designated arity.
We use $f^{(r)}\in\Sigma[L]$ to say that $f$ is
in the \emph{signature} of $L$ and has arity $r$.

\section{Term Rewriting Systems}
\label{prelim:term-rewriting}

Term rewriting systems are fundamental in automated reasoning, program transformation, and symbolic computation. 
They provide a formal framework for manipulating symbolic expressions, allowing us to systematically transform complex expressions into simpler or more useful forms. 
In the context of this thesis, equality saturation builds upon and enhances term rewriting systems, while maintaining similar underlying semantics, thus we briefly formalize rewriting systems here.

We assume a formal language $\Lang$ of terms over some vocabulary of symbols. A rewrite rule is a universally quantified formula of the form:

\[ \forall x_1, \ldots, x_n. t_1 \rwto t_2 \]

We use the notation $\Rewrite = t_1 \rwto t_2$ as a shorthand, with the understanding that all free variables are implicitly quantified.
For a semantic equality law $t_1 = t_2$, we typically assume both $t_1 \rwto t_2$ and $t_2 \rwto t_1$ exist as rewrite rules. 
Note, in the context of this thesis, we assume that all terms in rewrite rules are closed under the implicit universal quantification, i.e., there are no existential quantifiers in rewrite rules.
This of course is not always true, but in this context we do not support rewriting with existential quantifiers due to its complexity, hence it is just treated as some uninterpreted function symbol.

The application of a rewrite rule $\Rewrite = t_1 \rwto t_2$ to a term $s$ is defined as a relation $\RewriteRelation{}{}$ in terms of contexts and substitutions:

\begin{definition}\label{screening:rw-equiv}
Given a rewrite rule $\Rewrite = t_1 \rwto t_2$, we define a corresponding relation
$\RewriteRelation{}{}$ such that $\RewriteRelation{s_1}{s_2} \iff s_1=C[t_1\sigma] \land s_2=C[t_2\sigma]$
for some context $C$ and substitution $\sigma$ for the universally quantified variables of $t_1, t_2$.
where:
\begin{itemize}
    \item A \emph{context} $C$ is a term with a single hole, and $C[t]$ denotes the term obtained by filling the hole with $t$.
    \item A \emph{substitution} $\sigma$ is a mapping from variables to terms, applied to replace variables in an expression.
\end{itemize}
\end{definition}

We define equivalence of terms when a sequence of rewrites can identify the terms in either direction.
This slight asymmetry is what motivates the following definitions.

\begin{definition}\label{screening:symmetric}
Given a relation $\RewriteRelation{}{}$ we define its symmetric closure:

\centering
$\SRewriteRelation{t_1}{t_2} \iff \RewriteRelation{t_1}{t_2} ~\lor~ \RewriteRelation{t_2}{t_1}$
\end{definition}

\begin{definition}\label{screening:rw-set-equiv}
Given a set of rewrite rules $G_\Rewrite=\RewriteSys$, we define a relation as union of the relations of the rewrites: $\RewriteSysRelation{}{} \eqdef \bigcup_{i}\SRewriteRelationi{}{}$.\\
In the sequel, we will mostly use its reflexive transitive closure, $\TRewriteSysRelationSymbol$.
\end{definition}

The relation $\TRewriteSysRelationSymbol$ is reflexive, transitive, and symmetric,
so it is an equivalence relation over $\Lang$.
Under the assumption that all rewrite rules in $\RewriteSys$ are semantics preserving, for any equivalence class $[t] \in \Lang \big/ {\TRewriteSysRelationSymbol}\!$, all terms belonging to $[t]$ are definitionally equal.
We present a small example of how a term might be rewritten on \autoref{perlim:rewriting_example}.
Since $\Lang$ may be infinite, it is essentially impossible to compute $\TRewriteSysRelationSymbol$ completely. 
Any practical algorithm can only explore a finite subset $\Terms \subseteq \Lang$, and in turn, construct a subset of $\TRewriteSysRelationSymbol$. 
This limitation motivates the need for more efficient representations and manipulation of term equivalences, which we will address in the following sections on e-graphs and equality saturation.

\begin{example}[Arithmetic Expression Rewriting Example]
\label{perlim:rewriting_example}
Consider a simple arithmetic language with addition and multiplication. We define the following rewrite rules:

\begin{align*}
R_1:& \quad \forall x.~x + 0 \rwto x \\
R_2:& \quad \forall x, y.~x + y \rwto y + x \\
R_3:& \quad \forall x.~x \times 1 \rwto x \\
R_4:& \quad \forall x, y, z.~x \times (y + z) \rwto (x \times y) + (x \times z)
\end{align*}

Now, let's apply these rules to the term $t = 2 \times (3 + 0)$:

\begin{align*}
2 \times (3 + 0) &\RewriteRelationR{R_4} (2 \times 3) + (2 \times 0) &\text{(applying $R_4$ with $x = 2, y = 3, z = 0$)} \\
                 &\RewriteRelationR{R_3} (2 \times 3) + 0 &\text{(applying $R_3$ to the right term)} \\
                 &\RewriteRelationR{R_1} 2 \times 3 &\text{(applying $R_1$ to the whole expression)}
\end{align*}

Alternatively, we could apply $R_1$ first:

\begin{align*}
2 \times (3 + 0) &\RewriteRelationR{R_1} 2 \times 3 &\text{(applying $R_1$ with $x = 3$)} \\
                 &\text{(no more applicable rewrites)} \\
\end{align*}

It's important to note that some rules, like $R_2$ (commutativity), could lead to non-termination if applied repeatedly.
\end{example}

\section{E-graphs}
\label{perlims:e-graphs}

E-Graphs, short for equality graphs, are data structures that efficiently represent sets of equivalent expressions. 
Originally developed for use in automated theorem proving \cite{egraphsnelson1980fast}, E-Graphs have found applications in various domains, including program optimization \cite{eqsat}, program synthesis \cite{herbie}, and theory exploration \cite{thesy}.

\begin{definition}[E-graph]
An e-graph $\mathcal{G}$ over a language $\Lang$ with signature $\Sigma$ is a tuple $\mathcal{G} = (\Lang, S, \cong)$, where:
\begin{itemize}
    \item $\Lang$ is the set of all possible terms in the language.
    \item $S \subseteq \Lang$ is the set of terms explicitly added to the e-graph.
    \item $\cong$ is a congruence relation over $\Lang$.
\end{itemize}
We say that a term $t \in \Lang$ is "contained" in the e-graph if $t \in S$.
\end{definition}

\begin{definition}[Congruence Relation]
A congruence relation $\cong$ over a language $\Lang$ with signature $\Sigma$ is an equivalence relation that is closed under the function symbols in $\Sigma[L]$.
This property is known as \emph{congruence closure}. 
Formally, for any $f^n \in \Sigma[L]$ and terms $t_1, \ldots, t_n, s_1, \ldots, s_n \in \Lang$:
\[
(\forall i \leq n.~t_i~\cong~s_i) \implies f(t_1, \ldots, t_n)~\cong~f(s_1, \ldots, s_n)
\]
\end{definition}

The maintenance of congruence closure in e-graph as an invariant significantly influences the design and implementation of e-graph actions.
The egg library~\cite{egg} revolutionizes the application of e-graphs by explicitly supporting the equality saturation workflow.
It enables the periodic maintenance of congruence closure, via \emph{deferred rebuild}, allowing for the amortization of associated rebuilding costs.

In egg, the authors present the e-graph as a union-find-like data structure, augmented to support operations on expressions.
This implementation is primarily achieved through the utilization of three key structures: a hash-cons table, a union-find structure, and an e-class map.
These structures collectively underpin the functionalities integral to the operation of the e-graph.

\begin{enumerate}
%
% Union find
\item The \underline{union-find} component is responsible for keeping track of merged e-classes and maps each e-class id to a single representative for all (transitively) merged e-classes.
This information is later used to canonicalize the keys and values of the hash-cons. It essentially represents the equivalence relation part of $\cong$.
%
% E-Class map
\item The \underline{e-class map} stores the structure of the e-graph.
For each e-class id, the map keeps all the e-nodes that are contained therein.
E-nodes are similar to AST nodes except that their children point to e-class ids instead of containing a single sub-term each. 
That is, an e-node represents an application $f(x_1,\ldots,x_n)$ for $f^n \in \Sigma[L]$ where each $x_i$ represents an e-class id.
% 
% Hash-Cons
\item The \underline{hash-cons} table maps e-nodes to their containing e-class id.
An important aspect of the hash-cons is that after rebuilding, its keys and values are expected to be \emph{canonical}. 
That is, whenever e-classes are merged one of their ids becomes ``the'' representative.
\end{enumerate}

An e-class with id $e$ represents a set of terms defined recursively as:
\begin{align*}
L(e) &= \{f(t_1,..,t_k)~|~ \\ 
     & \qquad f(e_1,..,e_k)\in M(e), t_i\in L(e_i)\mbox{~for~}i=1..k\}
\end{align*}
We will use the notation $[t]$ to refer to e-class id where $t\in L([t])$.

\begin{example}
The terms $\tmax(x, y)$ and $x - y$ are both represented in the e-graph in \autoref{overview:egraph-max}(a) using e-classes $\ecid5$ and $\ecid6$, respectively, with  the following e-nodes:
%
\[
M \quad = \quad
\adjustbox{valign=t}{
$
\begin{array}{@{}ll}
  \ecid1 \mapsto \{\ttrue\} & \ecid2 \mapsto \{\tfalse\} \\
  \ecid3 \mapsto \{x\} & {} \ecid4 \mapsto \{y\} \\
  \ecid5 \mapsto \{\tmax(\ecid3, \ecid4)\} & \ecid6 \mapsto \{\ecid3 - \ecid4\}
\end{array}
$
}
\]

\begin{comment}
The e-graph maintains a union-find, which, in this trivial example, is a bit boring since it is an identity relation.

The contents of the hash-cons (which can be easily discerned by inverting $M$) are:
\[H \quad = \quad a \mapsto \ecid1 \quad b\mapsto \ecid2 \quad 
   c\mapsto \ecid3 \qquad
  \ecid2 + \ecid3 \mapsto \ecid4 \qquad \ecid1 \cdot \ecid4\mapsto \ecid5\]
\end{comment}

\end{example}

An e-graph where every e-class is a singleton, like this one, is just a forest of expression trees with sharing.
The situation becomes more interesting once we start mutating the graph via its dedicated operations.

% Functionalities
\begin{enumerate}
% (1) Insert
\item \underline{Insert} - Adds a term $t$ to the e-graph, one e-class per AST node, reusing e-classes where possible by searching the hash-cons.
% (2) Union
\item \underline{Merge} - Merging two e-classes by applying a union operation of the union-find and merging the classes in the e-class map.
This, however, temporarily invalidates the invariant of the hash-cons and e-class map that all e-class ids and e-nodes must be canonical.
% (3) Congruence closure
\item \underline{Rebuilding (Congruence closure)} - As explained before, a union of $[x]$ into $[y]$ necessitates replacing any e-node $f([x],[z])$ by $f([y],[z])$.
Moreover, if $f([x],[z])\in[w_1], f([y],[z])\in[w_2]$,
then, following this replacement, both $[w_1]$ and $[w_2]$ now contain
$f([y],[z])$, meaning that $[w_1] = [w_2]$ and evoking a cascading union of $[w_1], [w_2]$.
A significant contribution by egg is the concept of deferred (and thus periodic) rebuilding.
This periodic rebuilding is highly efficient and well-suited for equality saturation.
% (4) E-matching
\item  \underline{E-matching} - Looking up a \emph{pattern} in the set of terms represented by the e-graph in a top-down manner, traversing the e-nodes downward via the e-class map.
A pattern is a term with (zero or more) \emph{holes} represented by metavariables $?v_{1..k}$.
When talking about formulas from universally quantified formulas, the universally bound variables are treated as metavariables.
For example, $(?v_1+1)\cdot ?v_2$ is a pattern.
Pattern lookup is important for rewriting in equality saturation.
\end{enumerate}

\section{Equality Saturation}

Equality Saturation is a technique that leverages e-graphs to perform exhaustive application of rewrite rules without relying on heuristics to choose which rewrites to prefer or committing to any particular sequence of rewrites.

\subsection{Rewriting}
We assume a background set of symbolic \emph{rewrite rules} (r.r.), each of the form $t\rwto s$,
where $t$ and $s$ are patterns as explained in e-matching operation item above.
A \emph{match} $\sigma$ of pattern $t$ on the e-graph, is an assignment mapping metavariables to e-class ids.
$t\sigma$ represents an e-node, and we will denote its equality class as $[t\sigma]$.
Applying the r.r. is done by merging the e-classes $[s\sigma]$ and $[t\sigma]$. 
Because the e-node $s\sigma$ might not be in $S$ (i.e. it is new), it needs to also be inserted, resulting in $union([t\sigma], insert(s\sigma))$.
Repetitively applying such rewrite rules to a set of terms can be used to generate growing sets of terms that are equivalent, according to rewrite semantics, to ones in the starting set.
Ideally, the set eventually \emph{saturates}, in which case the e-graph now describes \emph{all} the
terms that are rewrite-equivalent.
We point out that in many situations, the e-graph keeps growing as a result of rewrites and never gets saturated---so the number of successive rewrite iterations, or ``rewrite depth'', has to be bounded.

A \emph{conditional rewrite rule} (c.r.r.)~\cite{jcss/Bergstra} is a natural extension of a r.r. that has the following form:
$\varphi \Rightarrow t \rwto s$
where $\varphi$ is a precondition for rewriting $t$ to $s$. For example, the rules for $max$ are:
${?x} > {?y} \Rightarrow \tmax({?x}, {?y}) \rwto {?x}$ and 
${?x} \leq {?y} \Rightarrow \tmax({?x}, {?y}) \rwto {?y}$.
%
The semantics of a precondition $\varphi$ is defined such that a term matching the pattern of $\varphi$
must be unified with Boolean $\ttrue$ in order for the rewrite to be applied.

\subsection{The Equality Saturation Process}

The equality saturation process, as implemented in the egg library \cite{egg}, operates in iterations. Each iteration consists of three main steps:

\begin{enumerate}
    \item \textbf{Matching:} All rewrite rules are matched against the current e-graph, collecting substitutions $\sigma$ for each rule. 
    Importantly, the e-graph is not updated during this phase.
    \item \textbf{Applying:} All collected matches are applied to the e-graph. 
    This step adds new terms to the e-graph and updates the equality relation.
    \item \textbf{Rebuilding:} The e-graph is rebuilt to restore the congruence closure property.
\end{enumerate}

This process continues until a fixed point is reached (no new terms or equalities are added), or until a predefined resource limit (e.g., number of iterations, time, or memory usage) is met.
The equality saturation process effectively constructs a subset of the rewrite relation $\TRewriteSysRelationSymbol$. 
Each iteration expands this subset by adding new terms and equalities to the e-graph.
Formally, if we denote the e-graph after the $i$-th iteration as $\mathcal{G}_i = (\Lang, S_i, \cong_i)$, then:

\begin{itemize}
    \item $S_i \subseteq S_{i+1}$ (the set of terms grows monotonically)
    \item $\mathcal{\cong}_i \subseteq \mathcal{\cong}_{i+1}$ (the equality relation becomes finer)
    \item For any terms $t_1, t_2 \in S_i$, if $t_1 \mathcal{\cong}_i t_2$, then $t_1 \TRewriteSysRelation{} t_2$
\end{itemize}

The key advantage of this approach is that it allows for the efficient application of many rewrite rules without the need to rebuild the e-graph after each individual rule application. This is particularly beneficial when dealing with large sets of rewrite rules or when exploring many possible rewrite sequences.

\begin{comment}
A preliminaries chapter is not necessary, but it may be a good idea to use it for presenting your theoretical/mathematical framework in a more detailed and technical way than the introduction, and to perhaps establish some basic lemmata/observations common to multiple chapters of your thesis.

\section{Some section}

Let's define some concept we'll be using throughout the thesis.

\begin{definition}
The \emph{von Neumann model} of a computer, also known as the \emph{Princeton architecture} is an architecture for digital computers, which consists of a processing units, containing an ALU and processing registers; a control unit consisting of an instruction register and a program counter; a memory unit which stores both data and instructions; and input-and-output mechanisms.
\end{definition}

\section{Acronyms and abbreviations}

Your thesis will typically have a set of significant terms, abbreviations and acronyms. Technion guidelines mandate that you place a list of these at the beginning of your thesis; and that they be defined upon first use. And, indeed, if you followed read this sample thesis carefuly thus far you should have seen ``\nameref{chap:notation-and-abbreviations}'' following the abstract.

When writing your thesis, collect such terms and their corresponding definitions in the \texttt{front/abbrevs.tex} file --- using the commands \verb|\newacronym|,  \verb|\newabbreviation| and \verb|\newglossaryentry|; the latter command is used for symbols and short, but unabbreviated, terms.

In the body of your thesis, your first use of a term will typically be where you want to also include the text of its definition. You don't need to repeat the definition you've already entered! Let's explain with an example: You've defined the term \gls{DIY} beforehand; when using it, you invoke the command \verb|\gls{DIY}|.\footnote{The 
\texttt{\textbackslash{}gls} command originates in the \texttt{glossaries-extra} package, which is used to automate the handling of notation \& abbreviations.} This command does several things:
\begin{itemize}
	\item It ensures the term \gls{DIY} is included in the list of Notation \& Abbreviation, at the beginning of the thesis; the entry for the term will also include the page on which it first appears;
	\item It produces the definition text; and finally
	\item It adds the defined term --- \gls{DIY} --- in parentheses, after the definition.
\end{itemize}
In later invocations of \verb|\gls{DIY}|, only the short form (\gls{DIY}) will be printed, not the definition, and no parentheses. (This also means that if you move text around in your thesis you don't have to worry about defining on first use - that's already taken care of.) 
\end{comment}