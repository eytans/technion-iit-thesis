\chapter{Conclusion and Future Directions}
\label{chap:conclusion}

This thesis has explored the application of equality saturation techniques to enhance automatic deductive reasoning across various domains.
We have developed novel approaches that address key challenges in this field.

The first basic challenge we faced when attempting to create automatic deduction tools, was incomplete background theory. 
Deductive reasoning is limited by the background theory, and thus the problem of an incomplete one needed addressing.
TheSy (presented in \autoref{chap:thesy}), our symbolic theory exploration system, represents a significant advancement in the field of theory exploration.
By leveraging equality saturation techniques, TheSy efficiently discovers and proves new lemmas in mathematical theories. 

Our experiments demonstrated that TheSy can discover a broader knowledge base than previous testing-based approaches while being faster in most cases. 
This efficiency gain is particularly pronounced when dealing with complex data types and function compositions, where concrete testing methods often struggle.

One significant issue encountered with TheSy was trying to perform conditional reasoning in combination with equality saturation.
Existing solutions provide either mechanisms to have some local conditional reasoning, or create a copy of the underlying e-graph leading to higher memory consumption.
To address this challenge within the equality saturation framework, we introduced colored e-graphs (presented in \autoref{chap:colors}).
This novel extension to e-graphs provides a memory-efficient and flexible approach to representing multiple equality relations within a single structure. Key features of colored e-graphs include:
A layered approach to representing multiple congruence relations, with each layer corresponding to a specific condition or assumption;
efficient sharing of common substructures, reducing memory overhead;
specialized algorithms for merging, rebuilding, and e-matching operations that take advantage of the layered structure.

Our evaluation showed that Colored E-graphs can significantly reduce memory usage compared to maintaining separate e-graphs for different conditions, while maintaining similar runtime performance. 
This advancement opens new possibilities for applying equality saturation techniques to problems involving complex conditional reasoning.

Even though we presented solutions for basic problems encountered in automated deductive reasoning, and specifically in equality saturation, the challenge of applying equality saturation as a theorem prover remained.
The main advantage of the equality saturation approach is in its ability to deal with a huge background theory, but it suffers from no other builtin reasoning techniques, limiting it to the said background theory.
Thus, we developed the Lightweight Equality Saturation (LES) prover in the domain of interactive theorem proving (presented in \autoref{chap:les}).
In the context of interactive theorem proving, applying automatic theorem provers is considered difficult due to the vast amount of available lemmas and theorems.
LES bridges the gap between automated reasoning and interactive proof assistants by providing a fast, automated reasoning tool that integrates well with interactive environments. 

Our experiments with LES showed promising results in automatically proving theorems within the Coq proof assistant, demonstrating its potential to enhance the capabilities of ITPs.
Even though, there are still many open directions to enhance it's capabilities.

\section{Open Questions and Future Directions}

While our work has made significant strides in applying equality saturation to automatic deductive reasoning, several important questions and research directions remain open for future exploration:

A key challenge for future research is to extend symbolic theory exploration to efficiently discover conditional lemmas. 
For example, can we use colored e-graphs to find lemmas such as a sorted list is increasing? 
This presents challenges in both scale and discovery of conditions and lemmas.
The potential of colored e-graphs opens up several directions for exploring their effectiveness in other tasks, such as global program optimizations or symbolic execution, hopefully leading to new state-of-the-art discoveries.

Although LES is an effective prover, several direction for future studies remain. 
Extending it with conditional and inductive reasoning using colored e-graphs could significantly improve its ability.
Also, due to it's interactive design, LES has the potential to serve as a deductive reasoning framework for agents, whether human or artificial, that would direct difficult to automate procedures such as case splitting, induction, and existential variables instantiation.

\section{Closing Remarks}

The work presented in this thesis address key challenges in theory exploration, conditional reasoning, and integration with interactive theorem provers, and have opened new avenues for research in automated reasoning.

The three components presented, TheSy, colored e-graphs, and LES, synergize to create a thorough approach to improving the applicability of automatic deductive reasoning.
TheSy's ability to efficiently discover new lemmas provides a richer knowledge base, which LES can leverage to handle complex reasoning scenarios.
Also, colored e-graphs for conditional reasoning can potentially imporve theory exploration systems and theorem provers based on equality saturation. 
These innovations form a potent toolkit that addresses key challenges in automated reasoning.

We hope these insights and innovations will further advance automatic deductive reasoning, bringing creating effective and autonomous reasoning agents that will help in a wide range of domains.